\documentclass[8pt]{beamer}
%epackage[french]{babel}
\usepackage[latin1]{inputenc}
\usepackage{listings}
\usepackage{times}
\usepackage{wasysym}
\usepackage[T1]{fontenc}


\definecolor{mongris}{gray}{0.8}           % definition couleur grise
\newcommand{\dd}{\footnotesize $\Diamond$}

\newcommand{\HH}{ \vspace{0.5pt}\hrule}
\newcommand{\round}[1]{\lceil #1 \rfloor}  % notation arrondi
\def\eme{$^{\textrm{{\`e}me}}$}                  % i {\`e}me
\def\num{n^{\circ}}                        % numero
\def\Num{N^{\circ}}                        % Numero
\def\sinc{\mathrm{sinc}}                   % sinus cardinal
\def\ere{$^{\textrm{{\`e}re}}$}                % {\`e}re
\def\er{$^{\textrm{{e}r}}$}                % {\`e}re
\def\eg{\emph{e.g.} }                      % e.g.
\def\ie{\emph{i.e.} }                      % i.e.
\def\etc{\emph{etc}}                       % etc
\def\cm{\,cm}                              % cm
\def\met{\,m}                              % m
\def\mm{\,mm}                              % mm
\def\deg{$^\circ$}                         % degres
\def\ud{\mathrm{d}}                        % pour dx dy ...


\def \R {{\Bbb R}}
\def \I {{\Bbb I}}
\def \H{{\Bbb H}}
\def \F {{\Bbb F}}
\def \S {{\Bbb S}}
\def \B {{\Bbb B}}
\def \Z {{\mathbb Z}}
\def \G {{\mathbb G}}
\def \L {{\mathcal{L}}}
\def \C {{\mathcal C}}
\def \P {{\mathcal P}}
\def \Q {{\mathcal Q}} 
\def \E{{\mathcal E}}
\def \D{{\mathcal D}}
\definecolor{mybluecolor}{RGB}{116,121,149}

\newcommand{\darky}[1]{{\usebeamercolor[fg]{block title example} #1}}
\newcommand{\myblue}[1]{{\color{mybluecolor}\aut{[#1]}}}

\newcommand{\ball}  {\ensuremath{B}}
\newcommand{\AMDR}{\operatorname{AMD}}
\newcommand{\AMD}{\operatorname{AMD}}

\newcommand{\MAset}{\ensuremath{\mathrm{A\!M}} }
\newcommand{\MAsetg}{\ensuremath{\MAset^g } }

\def \PS {{\aut{Planar-4-3-SAT}}}
\def \R {{\Bbb R}}
\def \I {{\Bbb I}}
\def \F {{\Bbb F}}
\def \S {{\Bbb S}}
\def \Z {{\mathbb Z}}
\def \L {{\mathcal{L}}}
\def \C {{\mathcal C}}
\def \P {{\mathcal P}}
\def \Q {{\mathcal Q}} 
\def \E{{\mathcal E}}
\def \D{{\mathcal D}}
\def \BD {{\bar{\mathcal{D}}}}
\def \etal {{\it et al.~}}
\def\arc{\mbox{arc}}
\definecolor{mongris}{gray}{0.8}          
\newcommand{\fup}[1]{\uparrow#1\uparrow}
\newcommand{\fdown}[1]{\downarrow#1\downarrow}
\newcommand{\sI}[1]{\overline{\tt #1}}
\newcommand{\iI}[1]{\underline{\tt #1}}
\newcommand{\e}[5]{#1 & #2 & #3 & #4 & #5 \\}
\newcommand{\eh}[5]{\text{#1} & \text{#2} &  \text{#3} &  \text{#4} & \text{#5}\\} 

\usepackage{beamerthemeliris2}
\useoutertheme{smoothbars}

\title[DGtal Meeting - Kernel Package Introduction]{DGtal: Introduction to
  DGtal Kernel}
\subtitle{\url{http://liris.cnrs.fr/dgtal}}

\author{D. Coeurjolly}
%\author[DGtal~~~~~~~~~~~~~~~~~~~~~~~~~~David Coeurjolly]{David Coeurjolly}


 \newcommand{\fod}[2]{\multicolumn{2}{p{3.5cm}}{\emph{#1}\dotfill} &
      \multicolumn{2}{p{9cm}}{#2}\\}
    \newcommand{\fodt}[4]{\emph{#1} & {\footnotesize \textsl{#2}} & #3 & \small #4\\}
    % \newenvironment{ta}{\begin{tabular}{p{3.5cm}p{9cm}}}{\end{tabular}\\}
    \newenvironment{ta}{\begin{tabular}{crll}}{\end{tabular}\\}
    % \vfill


\newcommand{\aut}[1]{{\sc #1}}             % auteur en small capsu


%\institute%[XXX]
%{%
%
%  {\bf Laboratoire d'InfoRmatique en Image et Syst�mes d'information} \\
%  { \scriptsize{
%  LIRIS UMR 5205 CNRS/INSA de Lyon/Universit� Claude Bernard Lyon 1/Universit� Lumi�%re Lyon 2/Ecole Centrale de Lyon\\
%  INSA de Lyon, b�timent J. Verne\\
%  20, Avenue Albert Einstein - 69622 Villeurbanne cedex\\
%  \url{http://liris.cnrs.fr}}
%  }
%}



\graphicspath{{./Figures/}, {./../images/},{./Fig/}, {./ICPR2010/},{./Antoine/images/}; {./Images/}}


\begin{document}

\small






\begin{frame}[plain]
  \titlepage
\end{frame}



%\begin{frame}
%  \frametitle{Table of Contents}
%  \tableofcontents
%\end{frame}


\begin{frame}
  \frametitle{Package description}

  \begin{block}{Should contain\HH}
    \begin{itemize}
    \item Fundamental objects and methods to define a topological and geometric structure on $\Z^d$
    \end{itemize}
  \end{block}

  \begin{exampleblock}{Examples\HH}
    \begin{itemize}
    \item Digital space  and domains definitions
    \item Integer types (unitary ring)
    \item Point \& Vector
    \item Linear Algebra
    \item Digital sets
    \item \ldots
    \end{itemize}
  \end{exampleblock}

  \begin{alertblock}{Location\HH}
    \begin{itemize}
    \item
      \texttt{\{DGtal\}$\backslash$src$\backslash$DGtal$\backslash$kernel$\backslash$}
   \item   \texttt{\{DGtal\}$\backslash$tests$\backslash$DGtal$\backslash$kernel$\backslash$}
    \end{itemize}
  \end{alertblock}


\end{frame}
\lstset{
	language=[Visual]C++,
	keywordstyle=\bfseries\ttfamily\color[rgb]{0,0,1},
	identifierstyle=\ttfamily,
	commentstyle=\color[rgb]{0.133,0.645,0.133}\textit,
	stringstyle=\ttfamily\color[rgb]{0.627,0.126,0.941},
	showstringspaces=false,
	basicstyle=\ttfamily,
	numberstyle=\color[rgb]{0.2,0.2,0.2}\tiny\ttfamily,
	numbers=left,
	stepnumber=1,
        frame=single,
        framexleftmargin=13mm, 
        xleftmargin=12mm,
	numbersep=10pt,
	tabsize=2,
	breaklines=true,
	prebreak = \raisebox{0ex}[0ex][0ex]{\ensuremath{\hookleftarrow}},
	breakatwhitespace=false,
	aboveskip={1.5\baselineskip},
  columns=fixed,
  upquote=true,
  extendedchars=true,
% frame=single,
% backgroundcolor=\color{lbcolor},
}
%\lstset{language=c++, numbers=left, tabsize=2, frame=single, breaklines=true, basicstyle=\ttfamily,
%  numberstyle=\tiny\ttfamily, framexleftmargin=13mm, xleftmargin=12mm,keywordstyle=\color{black}\bfseries,commentstyle=\color{red}\textit}

\begin{frame}[containsverbatim]
  \frametitle{DGtal code skeleton}

Things to do
  \begin{enumerate}
  \item Fix the dimension
  \item Fix the Integer type (commutative ring (+,-,*))
  \item Define the digital space DGtal::SpaceND
  \end{enumerate}


  
\begin{lstlisting}
  #include "DGtal/base/Common.h"
  #include "DGtal/kernel/SpaceND.h"
  {...}
  typedef DGtal::int32_t Integer;
  typedef DGtal::SpaceND<6, Integer> Space6;
  
  typedef mpz_class IntegerGMP;
  typedef DGtal::SpaceND<6, IntegerGMP> Space6GMP;
\end{lstlisting}

Q: what's wrong with ?
\begin{lstlisting}
typedef DGtal::SpaceND<2, unsigned char> MySpaceUChar;
\end{lstlisting}
\end{frame}


\begin{frame}[containsverbatim]
  \frametitle{[DETAILS] Concept \& Models}
  
  
  \begin{alertblock}{Answer\HH}
    \texttt{unsigned char} does not define a ring !
  \end{alertblock}

\vspace{0.5cm}

\begin{block}{}
  Constraints on types and template parameters are defined with {\bf Concepts}
\end{block}

\vspace{0.3cm}

{\tt Integer} in {\tt SpaceND} should be a model of {\tt
  DGtal::CCommutativeRing}.

\vspace{0.5cm}
Concept Checking  with {\tt boost}
\begin{lstlisting}
  ...
  //Integer must be signed to characterize a ring.
  BOOST_CONCEPT_ASSERT(( CCommutativeRing<TInteger> ) );
  ...
\end{lstlisting}


\end{frame}

\begin{frame}[containsverbatim]
  \frametitle{Digital Space}

Types
    \begin{lstlisting}
      Integer
      Point
      Vector
      RealPoint
      RealVector
      Subspace
      Subcospace
    \end{lstlisting}
    +
    \begin{lstlisting}
      static Dimension dimension
    \end{lstlisting}
\end{frame}

\begin{frame}[containsverbatim]
  \frametitle{Point/Vector...}

Point/Vector in a $d-$Dimensional DGtal space.

\begin{itemize}
\item arithmetic operators (*, -, ...)
\item comparison operators (< ,>, ...)
\item methods associate to the canonical lattice associated to points (inf, sup, isLower,...)
\item  methods to compute various norms of Points/Vectors.
\end{itemize}

E.g.

\begin{lstlisting}
  #include "DGtal/base/Common.h"
  #include "DGtal/kernel/SpaceND.h"
  {...}
  typedef DGtal::int32_t Integer;
  typedef DGtal::SpaceND<2, Integer> Space2;
  typedef Space2::Point Point2;
  
  Point2 p(12, -34);
  Point2 q(2, -2);
  if (p < q)
   ...
\end{lstlisting}

\end{frame}


\begin{frame}[containsverbatim]
  \frametitle{StdDefs.h: ``Standard'' digital spaces}

  Shortcuts with {\tt StdDefs.h} with namespaces {\tt Z2i} and {\tt Z3i}.
  
  \begin{lstlisting}
    #include "DGtal/base/Common.h"
    #include "DGtal/utils/StdDefs.h"
    {...}
    
    DGtal::Z2i::Point p(12, -34);
    DGtal::Z2i::Point q(2, -2);
    if (p < q)
    ...
  \end{lstlisting}

  \begin{lstlisting}
    #include "DGtal/base/Common.h"
    #include "DGtal/utils/StdDefs.h"
    {...}
    
    DGtal::Z3i::Point p(12, 2, -34);
    DGtal::Z3i::Point q(2, 0, -2);
    if (p < q)
    ...
  \end{lstlisting}
\end{frame}


\begin{frame}[containsverbatim]
  \frametitle{Domains}

  \begin{block}{Short description\HH}
    Defines a subset of $\Z^d$ which we are working on. 
    \begin{itemize}
    \item a domain is parametrized by a specific SpaceND type
    \item must implement various {\tt Iterators} to scan the domain
      points 
    \end{itemize}
  \end{block}

Example: 

\begin{lstlisting}
  #include "DGtal/base/Common.h"
  #include "DGtal/helpers/StdDefs.h"
  #include "DGtal/kernel/domains/HyperRectDomain.h"
  {...}

  typedef HyperRectDomain<Z2i::Space> MyDomain;
  Z2i::Point a(-3,-4);
  Z2i::Point b(10,4);
  MyDomain domain(a,b);
\end{lstlisting}

\begin{alertblock}{}
More details later...
\end{alertblock}

\end{frame}

\begin{frame}[containsverbatim]
  \frametitle{Digital Sets}

 \begin{block}{Short description\HH}
    Define sets points in a given domain.
    \begin{itemize}
    \item several types of container (STL vector, STL set,...) which
      can be selected via a {\tt DigitalSetSelector}
    \item must implement methods to add/remove points
    \item must implement {\tt Iterators} to scan the 
      points 
    \item ...
    \end{itemize}
  \end{block}

Example:
\begin{lstlisting}
  Z2i::Point p1(  -10, -10  );
  Z2i::Point p2(  10, 10  );
  Z2i::Domain domain( p1, p2 );
  typedef DigitalSetSelector < Domain, BIG_DS + HIGH_ITER_DS + HIGH_BEL_DS >::Type SpecificSet;
  SpecificSet mySet( domain );

  Z2i::Point c(  0, 0  );
  mySet.insert( c );
  Z2i::Point d(  5, 2  );
  mySet.insert( d );
\end{lstlisting}
\end{frame}

\begin{frame}
  \frametitle{Images in DGtal}

  \begin{block}{Idea\HH}
    Mapping between Domain points and and values\\
    Models are parametrized by a Domain type and a Value type\\
    IO with readers, writers and SVG/PDF exports
  \end{block}

\vspace{0.5cm}
Several image containers:
\begin{itemize}
\item ImageContainerBySTLVector: linearization of nD domains
\item ImageContainerBySTLMap: (point,value) map
\item ImageContainerByHashTree: hierarchical nD-tree with geometric
  hashing functions.
\end{itemize}

\vspace{0.5cm}
but also:
\begin{itemize}
\item ImageContainerByITKImage: use ITKImage in DGtal
\end{itemize}

\end{frame}


\begin{frame}[containsverbatim]
  \frametitle{Key ideas}

  \begin{alertblock}{Type Inclusion\HH}
    \begin{center}
      \{dimension, Integer\} $\rightarrow$  SpaceND  $\rightarrow$ Domain  $\rightarrow$ DigitalSet
     \end{center}
  \end{alertblock}

  \begin{alertblock}{Concept checking\HH}
  \end{alertblock}

 \begin{alertblock}{StdDefs\HH}
  \end{alertblock}


 \begin{alertblock}{Visualisation\HH}
   Kernel objects are DGtal stream compliant (2D and 3D).
  \end{alertblock}

e.g.
\scriptsize
\begin{lstlisting}
  Point a,b;...
  Domain domain(a,b);...
  DigitalSet set(....); ...
  
  Board2D board;
  board << domain
        << set
        << a
        << b;
        
  board.saveSVG("myset.svg")
\end{lstlisting}

\end{frame}



\end{document}



