
\section{Bilan et perspectives}

\begin{frame}
  \frametitle{Bilan}
 

 \begin{block}{D�finition des concepts (� discuter) } 
   \begin{itemize}
   \item courbe : liste (circulaire) de points (ordre)
		\begin{itemize}
			\item fournit un it�rateur sur les points (ou les vecteurs de d�placement entre deux points cons�cutifs ?)
		\end{itemize}
   \item primitive : courbe v�rifiant une propri�t� donn�e
		 \begin{itemize}
		 \item initialisation (un point plut�t que deux pour que la propri�t� soit toujours v�rifi�e) 
		 \item ajout � l'avant (un point, un vecteur d�placement ?), retourne V si la propri�t� reste vraie (et le point est ajout�), F sinon (et le point n'est pas ajout� pour que la propri�t� reste vraie) 
		 \item ...
%\only<1>{
%		 \item ...
%}
%\only<2>{
%		 \item retrait � l'arri�re s'il y a plus d'un point, la propri�t� reste toujours vraie
%}
		 \end{itemize}
   \item d�composition : liste (circulaire) de primitives recouvrant une courbe (ordre)
		\begin{itemize}
			\item fournit un it�rateur sur les primitives
%			\item mod�lise les relations entre les points de la courbe et les primitives ? 
		\end{itemize}
   \end{itemize}
 \end{block}


\end{frame}

\begin{frame}
  \frametitle{Perspectives}

 \begin{block}{Regarder ce qu'on a fait} 
   \begin{itemize}
   \item Am�liorations de \emph{FreemanChain}, \emph{ArithmeticalDSS}, \emph{GreedySegmentation} (si besoin)
   \item Ecriture des concepts de courbe, primitive et d�composition (si on est d'accord).        
   \end{itemize}
 \end{block}

 \begin{block}{Aller plus loin...} 
   \begin{itemize}
   \item D'autres courbes (8-connexes ?)
		 \begin{itemize}
		 \item discr�tisation de courbes euclidiennes
		 \item extraction du bord de r�gions connexes
		 \end{itemize}
   \item D'autres primitives (il y a le choix!)
   \item D'autres d�compositions (couverture)        
   \end{itemize}

 \end{block}

\end{frame}



