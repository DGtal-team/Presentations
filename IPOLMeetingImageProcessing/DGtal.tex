% please do not modify the document and font size
\documentclass[11pt, a4paper]{article}
\usepackage{a4wide}

% this is just a default list of packages, feel free to add any
% (standard) package you need
\usepackage{amssymb,amsmath}
\usepackage[mathletters]{ucs}
\usepackage[utf8x]{inputenc}
\usepackage[breaklinks=false,unicode=true,pdfborder={0 0 0}]{hyperref}
\usepackage{graphicx}


\def\DGtal{\texttt{DGtal} }
\date{}
% use the speaker names as authors of this document,
% this will help the audience identify the speakers
% use the software name as the title of this document
\title{\DGtal: Digital Geometry Tools and Algorithms Library}
\author{David Coeurjolly$^1$ \qquad \qquad Jacques-Olivier Lachaud$^2$ \\
  (For the DGtal Team)\\
${}^1$ LIRIS (UMR CNRS 5205), Universit\'{e} de Lyon, F-69622 \\
${}^2$ LAMA (UMR CNRS 5127), Universit\'{e} de Savoie, F-73376\\
}

\begin{document}
\maketitle

DGtal is a generic open source library for Digital Geometry
programming for which the main objective is to structure different
developments from the digital geometry and topology community. The
aims are numerous: make easier the appropriation of our tools for a
neophyte (new PhD students, researchers from other topics, …), permit
better comparisons from new methods with already existing approaches
and to construct a federative project. Another objective of DGtal is
to simplify the construction of demonstration tools to share new
results and potential efficiency of the proposed work.

\section*{Overview}

The main purpose of the library is to focus on algorithms and
data-structures from the digital geometry community. For short,
Digital Geometry can be simply characterised as a set of definitions,
theorems and algorithmic tools that deal with the topological and
geometric properties of subsets of digital pictures. Even if the
domain emerged during the second half of the 20th century with the
birth of computer graphics and digital image processing, many links
have been demonstrated between Digital Geometry results and
fundamental theorems in mathematics (arithmetic, geometry, topology,
\ldots), discrete mathematics (word theory, combinatorics, graph
theory,\ldots) or computer science (algorithmic, image processing,
\ldots).

Hence, \DGtal package decomposition follows these principles:
\begin{description}
  \item[Kernel Package]: This package contains core concepts, objects
    and methods which are used in other higher level packages. For
    example, it defines number types considered in DGtal, fundamental
    structures such as the digital space, the digital domain, as well
    as basic linear algebra tools.

  \item[Arithmetic Package]: This package gathers tools to perform
    arithmetic computations. Standard arithmetic computations are
    provided: greatest common divisor, Bézout vectors, continued
    fractions, convergent, intersection of integer
    half-spaces. Several representations of irreducible fractions are
    provided. They are based on the Stern-Brocot tree structure. With
    these fractions, amortized constant time operations are provided
    for computing reduced fractions. 



  \item[Topology Package]: This package contains core concepts,
    objects and methods which are used in other higher level
    packages. For example, it defines number types considered in
    DGtal, fundamental structures such as the digital space in
    arbitrary dimension, the digital domain, as well as basic linear
    algebra tools.

  \item[Geometry Package]: This package contains geometry related
    concepts, models and algorithms. It provides a generic framework
    for the segmentation of one-dimensional discrete structures, like
    strings, contours of 2d digital objects or nd digital curves. It
    also provides a generic framework for the estimation of
    geometrical quantities, either global, like length, or local, like
    normal or curvature. Several estimators are built from some well
    chosen segmentations. On the other hand, this package contains
    tools for the analysis of volumes of arbitrary dimension, by the
    means of separable and incremental distance transforms.


  \item[Image Package]: This package defines data-structures to
    represent generic images in arbitrary dimension. We also provide
    several image containers (linear vector, associative map,
    pointerless n-D octree-like images, ...). We use specific image
    containers to define bindings with other existing image processing
    libraries. For instance, a specific ITK image container allows
    \DGtal users to integrate Kiteware's Insight Toolkit filters in
    their \DGtal processing pipeline. 

  \item[IO Package]: In this package, we present DGtal tools and
    utilities to import/export images and visualize digital data using
    interactive (viewers) and non-interactive (boards) mechanisms.

  \item[Mathematical Package]: This package gathers various
    mathematical subpackages and modules. For now, it consists in a
    module for defining multivariate polynomials.



\end{description}


\section*{Key Features}

\section*{Implementation details}
 
On a more technical level, DGtal is developed in C++. Similarly to
other classical libraries such as CGAL, it follows the paradigm of
genericity with efficiency. This approach is made possible with the
now well-known notion of concepts, implemented with templated types.



\section*{Examples}

We would like to receive these 4 pages before Sunday 24th to have
enough time for printing.

\section*{Team \& Project Management}


\end{document}
