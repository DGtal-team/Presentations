\documentclass[pdftex,francais]{beamer}

% Copyright 2004 by Till Tantau <tantau@users.sourceforge.net>.
%
% This file can be redistributed and/or modified under
% the terms of the GNU Public License, version 2.

%% \ifx\themename\undefined
%%   \def\themename{default}
%% \fi

\usetheme{lama}
%\usetheme{Madrid}
%\usecolortheme{crane}

\usepackage{multirow}
\usepackage[latin1]{inputenc}           %%%  
\usepackage[T1]{fontenc}                %%%
\usepackage[francais]{babel}            %%%

\usepackage{multimedia}
\usepackage{hyperref}
\usepackage{tikz}
\usepackage{listings}

%\newtheorem{theorem}{Th�or�me}

%\setbeamercovered{transparent}

\title[Irreducible fractions, patterns and straightness in DGtal]{Irreducible fractions, patterns and straightness\\
  DGtal, Arithmetic package (since 0.5)
}

\author[J.-O. Lachaud]{Jacques-Olivier Lachaud}

\date{DGtal Meeting, june 2012}


\graphicspath{{Figures/},{Images/}}

%%% \AtBeginSection[]
%%% {
%%%   \begin{frame}<beamer>
%%%     \frametitle{Plan}
%%%     \tableofcontents[currentsection] %,currentsubsection]
%%%   \end{frame}
%%% }

\input{macros}

\definecolor{MyGreen}{rgb}{0,0.6,0}

\setbeamercolor{qcolorb}{fg={blue!20!black},bg={blue!15!white}}
\setbeamercolor{qcolorub}{fg={blue!10!black},bg={blue!30!white}}
\setbeamercolor{qcolorlb}{fg={blue!20!black},bg={blue!8!white}}
\setbeamercolor{qcolorulb}{fg={blue!10!black},bg={blue!40!white}}
\setbeamercolor{qcolorg}{fg={green!20!black},bg={green!15!white}}
\setbeamercolor{qcolorug}{fg={green!10!black},bg={green!30!white}}
\setbeamercolor{qcolorlg}{fg={green!20!black},bg={green!8!white}}
\setbeamercolor{qcolorulg}{fg={green!10!black},bg={green!40!white}}
\setbeamercolor{qcolorr}{fg={red!20!black},bg={red!15!white}}
\setbeamercolor{qcolorur}{fg={red!10!black},bg={red!30!white}}
\setbeamercolor{qcolorlr}{fg={red!20!black},bg={red!8!white}}
\setbeamercolor{qcolorulr}{fg={red!10!black},bg={red!40!white}}
\newenvironment{myblockbluish}[2]%
	       {\begin{beamerboxesrounded}[lower=qcolorb,upper=qcolorub,width=#1,shadow=true]{#2}}{\end{beamerboxesrounded}}
\newenvironment{myblocklbluish}[2]%
	       {\begin{beamerboxesrounded}[lower=qcolorlb,upper=qcolorulb,width=#1,shadow=true]{#2}}{\end{beamerboxesrounded}}
\newenvironment{myblockgreenish}[2]%
	       {\begin{beamerboxesrounded}[lower=qcolorg,upper=qcolorug,width=#1,shadow=true]{#2}}{\end{beamerboxesrounded}}
\newenvironment{myblocklgreenish}[2]%
	       {\begin{beamerboxesrounded}[lower=qcolorlg,upper=qcolorulg,width=#1,shadow=true]{#2}}{\end{beamerboxesrounded}}
\newenvironment{myblockredish}[2]%
	       {\begin{beamerboxesrounded}[lower=qcolorr,upper=qcolorur,width=#1
,shadow=true]{#2}}{\end{beamerboxesrounded}}
\newenvironment{myblocklredish}[2]%
	       {\begin{beamerboxesrounded}[lower=qcolorlr,upper=qcolorulr,width=#1,shadow=true]{#2}}{\end{beamerboxesrounded}}

%%%%%%%%%%%%%%%%%%%%%%%%%%%%%%%%%%%%%%%%%%%%%%%%%%%%%%%%%%%%%%%%%%%%%%%%%%%%%%%
%%%%%%%%%%%%%%%%%%%%%%%%%%%%%%%%%%%%%%%%%%%%%%%%%%%%%%%%%%%%%%%%%%%%%%%%%%%%%%%
%%%%%%%%%%%%%%%%%%%%%%%%%%%%%%%%%%%%%%%%%%%%%%%%%%%%%%%%%%%%%%%%%%%%%%%%%%%%%%%
  \lstset{language=c++, numbers=left, tabsize=2, frame=single, breaklines=true, basicstyle=\ttfamily\scriptsize,
     numberstyle=\tiny\ttfamily, framexleftmargin=13mm, xleftmargin=12mm,keywordstyle=\color{blue}\bfseries,%
     commentstyle=\sffamily\color{red}}

\begin{document}

\newlength{\unquart}
\setlength{\unquart}{0.21\textwidth}

%------------------------------------------------------------------------------
\begin{frame}
  \titlepage
\end{frame}
%------------------------------------------------------------------------------

%%%%%%%%%%%%%%%%%%%%%%%%%%%%%%%%%%%%%%%%%%%%%%%%%%%%%%%%%%%%%%%%%%%%%%%%%%%%%%%
\section{Introduction}
%%%%%%%%%%%%%%%%%%%%%%%%%%%%%%%%%%%%%%%%%%%%%%%%%%%%%%%%%%%%%%%%%%%%%%%%%%%%%%%

%------------------------------------------------------------------------------
\begin{frame}%[allowframebreaks]
  \frametitle{Arithmetic package content}

  \begin{myblocklbluish}{\textwidth}{Content (\alertred{New} package in DGtal 0.5)}
    \begin{itemize}
    \item elementary integer arithmetic algorithms (gcd, B�zout)
    \item several representations for irreducible fractions
      \begin{itemize}
      \item Stern-Brocot tree
      \item continued fractions
      \item rational approximations
      \end{itemize}
    \item patterns
    \item digital straight lines and subsegments
    \end{itemize}
  \end{myblocklbluish}

  \begin{myblocklredish}{\textwidth}{Location}
    \begin{itemize}
      \small
    \item \texttt{\{DGtal\}/src/DGtal/math/arithmetic}
    \item \texttt{\{DGtal\}/tests/math/arithmetic}
    \item \texttt{\{DGtal\}/examples/math/arithmetic}
    \end{itemize}
  \end{myblocklredish}

\end{frame}
%------------------------------------------------------------------------------

%%%%%%%%%%%%%%%%%%%%%%%%%%%%%%%%%%%%%%%%%%%%%%%%%%%%%%%%%%%%%%%%%%%%%%%%%%%%%%%
\section{Elementary arithmetic}
%%%%%%%%%%%%%%%%%%%%%%%%%%%%%%%%%%%%%%%%%%%%%%%%%%%%%%%%%%%%%%%%%%%%%%%%%%%%%%%

%------------------------------------------------------------------------------
\begin{frame}[fragile]%[allowframebreaks]
  \frametitle{Elementary arithmetic over arbitrary integer types}

  Class \href{http://liris.cnrs.fr/dgtal/doc/nightly/classDGtal_1_1IntegerComputer.html}{\texttt{\alertred{IntegerComputer}$<$Int$>$}}

  \begin{itemize}
  \item stores temporary variables (useful for {\tt BigInteger})
  \item elementary operations: max, min, abs, isPositive, ...
  \item provides classical arithmetic computations: gcd, extended
    Euclid, convergents, continued fraction
  \end{itemize}

  \lstset{language=c++, numbers=left, tabsize=2, frame=single, breaklines=true, basicstyle=\ttfamily\tiny,
     numberstyle=\tiny\ttfamily, framexleftmargin=13mm, xleftmargin=12mm,keywordstyle=\color{blue}\bfseries,%
     commentstyle=\sffamily\color{red}}
  \lstset{emph={gcd,extendedEuclid,getCFrac},emphstyle=\color{MyGreen}}

  \begin{lstlisting}
  typedef DGtal::BigInteger Integer;
  IntegerComputer<Integer> ic; // instance for computations
  Integer g = ic.gcd( 192, 128 ); // 64
  IntegerComputer<Integer>::Vector2I v
    = ic.extendedEuclid( 5, 12, 2 ); // solution to 5x+12y = 2
  std::cout << "5x+12y=2 <=> x=" << v[0] 
	    << " y=" << v[1] << std::endl; // 5x+12y=2 <=> x=10 y=-4
  std::vector<Integer> q; // quotients
  ic.getCFrac( q, 5, 13 ); // continued fraction
  std::cout << "5/13=[" << q[0] << ";" << q[1] 
	    << "," << q[2] << "," << q[3]
	    << "," << q[4] << "]" << std::endl; // 5/13=[0;2,1,1,2]
   \end{lstlisting}
  \begin{itemize}
  \item[$+$] more complex operations related to integer half spaces
  \end{itemize}
\end{frame}

%------------------------------------------------------------------------------
\begin{frame}[fragile]%[allowframebreaks]
  \frametitle{Elementary arithmetic over arbitrary integer types (II)}
  \lstset{language=c++, numbers=left, tabsize=2, frame=single, breaklines=true, basicstyle=\ttfamily\tiny,
     numberstyle=\tiny\ttfamily, framexleftmargin=13mm, xleftmargin=12mm,keywordstyle=\color{blue}\bfseries,%
     commentstyle=\sffamily\color{red}}
  \lstset{emph={gcd,extendedEuclid,getCFrac},emphstyle=\color{MyGreen}}
  \begin{lstlisting}
  ic.getCFrac( q, 
               Integer("51234567894643563456345635435722900123"), 
	       Integer("345678532087609239457759428901234" ) );
  std::cout << "[" << q[0];
  for ( unsigned int i = 1; i < q.size(); ++i )
    std::cout << "," << q[i];
  std::cout << "]" << std::endl;
  // [148214,2,29,3,3,1,1,1,8,2,5,1,4,3,4,1,1,2,1,1,2,1,1,2,1,5,1,1,
  //  4,2,2,1,1,1,4,3,2,1,1,2,3,1,2,3,1,14,1,3,13,7,1,1,1,1,1,9,1,1,
  //  27,3,1,1,3,8,1,1,1,8,6,1,1,2,6,3,1,1,4,4]
  \end{lstlisting}

\end{frame}

%%%%%%%%%%%%%%%%%%%%%%%%%%%%%%%%%%%%%%%%%%%%%%%%%%%%%%%%%%%%%%%%%%%%%%%%%%%%%%%
\section{Irreducible fractions}
%%%%%%%%%%%%%%%%%%%%%%%%%%%%%%%%%%%%%%%%%%%%%%%%%%%%%%%%%%%%%%%%%%%%%%%%%%%%%%%

%------------------------------------------------------------------------------
\begin{frame}[fragile]%[allowframebreaks]
  \frametitle{Properties of positive irreducible fractions}
  
  \begin{block}{Definition \alert{positive irreducible fraction}}
    \small A fraction $\frac{p}{q}$ with $p,q \in \Z^+, \gcd(p,q)=1$.
  \end{block}

  \begin{itemize}
  \item uniqueness, dense
  \item related to finite simple continued fractions (Euclid algorithm)
  \item generated by the Stern-Brocot tree
  \end{itemize}
\end{frame}
%------------------------------------------------------------------------------

%------------------------------------------------------------------------------
\begin{frame}[fragile]%[allowframebreaks]
  \frametitle{Simple continued fractions}

  \begin{block}{Definition \alert{simple continued fraction}}
    \scriptsize
    A number of the form $a_0+\cfrac{1}{a_1+\cfrac{1}{\ldots +
        \cfrac{1}{a_n}}}$, where $a_i$ are integers, commonly written
    as $[a_0;a_1,\ldots,a_n]$. The $a_i$ are the partial quotients.
  \end{block}
  \begin{itemize}
  \item \scriptsize Any simple continued fraction is a positive irreducible fraction
  \item \scriptsize Any positive irreducible fraction has two simple continued fraction representations
  \item \scriptsize use Euclid algorithm (\alertred{gcd}, \alert{quotients}), e.g. $\frac{5}{13}$.
    \scriptsize
    $\begin{array}{ccccccc}
       p & = & \alert{u} & * & q & + & r \\ \hline
       5 & = & \alert{0} & * & 13& + & 5 \\ 
      13 & = & \alert{2} & * & 5 & + & 3 \\
       5 & = & \alert{1} & * & 3 & + & 2 \\
       3 & = & \alert{1} & * & 2 & + & 1 \\
       2 & = & \alert{2} & * & \alertred{1} & + & 0 \\
  \end{array} \quad$ \raisebox{0.6cm}{$\frac{5}{13}=\alert{0}+\cfrac{1}{\alert{2}+\cfrac{1}{\alert{1}+\cfrac{1}{\alert{1}+\cfrac{1}{\alert{2}}}}}$}
  \end{itemize}
\end{frame}

%------------------------------------------------------------------------------
\begin{frame}[fragile]%[allowframebreaks]
  \frametitle{Convergents and approximation}

  \centerline{
    \includegraphics[width=0.7\textwidth]{approx-5-13}
  }

  \small
  {\color{blue} $z_4 = \frac{5}{13}=[0;2,1,1,2]$ }\\
  {\color{red} odd} convergents: 
  {\color{red} $z_3 = \frac{2}{5}=[0;2,1,1]$ }
  {\color{red} $z_1 = \frac{1}{2}=[0;2]$ }
  {\color{red} $z_{-1} = \frac{1}{0}=[]$ }\\
  {\color{green} even} convergents:
  {\color{green} $z_2 = \frac{1}{3}=[0;2,1]$ }
  {\color{green} $z_0 = \frac{0}{1}=[0]$ }

  \begin{itemize}
  \item convergents are the best approximations to fractions/real numbers
  \item thus related to digital straight lines
  \end{itemize}
\end{frame}
%------------------------------------------------------------------------------

%------------------------------------------------------------------------------
\begin{frame}[fragile]%[allowframebreaks]
  \frametitle{Stern-Brocot tree of irreducible fractions}

  \centerline{
    \includegraphics[width=0.8\textwidth]{Sternbrocot}
  }
  
  \begin{itemize}
  \item two starting fractions: $\frac{0}{1}$ and $\frac{1}{0}$
  \item mediant of two fractions: $\frac{p}{q} \oplus \frac{p'}{q'} = \frac{p+p'}{q+q'}$ (vector addition)
  \end{itemize}
  
\end{frame}


%------------------------------------------------------------------------------
\begin{frame}[fragile]%[allowframebreaks]
  \frametitle{Link with continued fractions}

  \centerline{
    \includegraphics[width=0.8\textwidth]{Sternbrocot}
  }

  \begin{itemize}
  \item $u_0, u_1,\ldots, u_k $ = sequence of Right-then-Left moves from $\frac{1}{1}$, except last (one less).
  \item e.g. $\frac{5}{13}=[0;2,1,1,2]$, thus $R^0L^2RLR^{\alertred{2-1}}$.
  \end{itemize}
\end{frame}
%------------------------------------------------------------------------------

%------------------------------------------------------------------------------
\begin{frame}[fragile]%[allowframebreaks]
  \frametitle{Useful operations on fractions}
  
  \only<1-3>{
    \begin{itemize}
    \item if we forget $+$, $-$, $*$, $/$ ..., interesting operations are related to the ``tree'' structure
    \item making a fraction from its quotients, getting quotients
    \item mediant, left or right descendant, adding a quotient
    \item father, previous partial, $m$-father,
    \item arbitrary convergent / reduced partial
    \end{itemize}
  }

  \only<2>{
    \begin{block}{Requirements}
    \begin{itemize}
    \item Perform these operations in quasi-constant time !
    \item But storing quotients cost $O(\log(\max(p,q)))$
    \end{itemize}
    \end{block}
  }

  \only<3>{
  \begin{myblocklbluish}{\textwidth}{Solution}
    \begin{itemize}
    \item irreducible fraction described by concept \href{http://liris.cnrs.fr/dgtal/doc/nightly/structDGtal_1_1CPositiveIrreducibleFraction.html}{\texttt{\alertred{CPositiveIrreducibleFraction}}}
    \item explicit representation of the Stern-Brocot tree
    \item each node stores $k, u_k, p_k, q_k$
    \item but on-the-fly instanciation of nodes.
    \end{itemize}
  \end{myblocklbluish}
}
\end{frame}
%------------------------------------------------------------------------------

%------------------------------------------------------------------------------
\begin{frame}[fragile]%[allowframebreaks]
  \frametitle{Models of irreducible fractions (I)}

    \centerline{
      \only<1>{\includegraphics[width=0.7\textwidth]{Sternbrocot-2}}%
      \only<2>{\includegraphics[width=0.7\textwidth]{Sternbrocot-nodes-5-13}}%
    }
  \small
  \begin{itemize}
  \item Class \href{http://liris.cnrs.fr/dgtal/doc/nightly/classDGtal_1_1SternBrocot.html}{\texttt{\alert{SternBrocot}}}, fraction is \texttt{SternBrocot::\alert{Fraction}}
  \item Each node knows 5 other nodes (fathers, reciprocal, direct descendants on demand)
  \item Simple, fast for small fractions, memory costly, operations in $O(u_k)$
  \end{itemize}


\end{frame}
%------------------------------------------------------------------------------

%------------------------------------------------------------------------------
\begin{frame}[fragile]%[allowframebreaks]
  \frametitle{Models of irreducible fractions (II)}

    \centerline{
      \only<1>{\includegraphics[width=0.7\textwidth]{LightSternbrocot}}%
      \only<2>{\includegraphics[width=0.7\textwidth]{LightSternbrocot-nodes-5-13}}%
    }
  \small
  \begin{itemize}
  \item Class \href{http://liris.cnrs.fr/dgtal/doc/nightly/classDGtal_1_1LightSternBrocot.html}{\texttt{\alert{LightSternBrocot}}}, fraction is \texttt{LightSternBrocot::\alert{Fraction}}
  \item Each node knows its reduced, mapping to next partials on demand
  \item fast for small fractions, less memory costly, but tricky cases
  \end{itemize}


\end{frame}
%------------------------------------------------------------------------------

%------------------------------------------------------------------------------
\begin{frame}[fragile]%[allowframebreaks]
  \frametitle{Models of irreducible fractions (III)}

    \centerline{
      \only<1>{\includegraphics[width=0.7\textwidth]{LighterSternbrocot}}%
      \only<2>{\includegraphics[width=0.7\textwidth]{LighterSternbrocot-nodes-5-13}}%
    }
  \small
  \begin{itemize}
  \item Class \href{http://liris.cnrs.fr/dgtal/doc/nightly/classDGtal_1_1LighterSternBrocot.html}{\texttt{\alert{LighterSternBrocot}}}, fraction is \texttt{LighterSternBrocot::\alert{Fraction}}
  \item Each node knows its origin, mapping to next partials on demand
  \item fast for big fractions, less memory costly, best trade-off
  \end{itemize}


\end{frame}
%------------------------------------------------------------------------------

%------------------------------------------------------------------------------
\begin{frame}[fragile]%[allowframebreaks]
  \frametitle{Using fractions}

  Creating fractions...

  \lstset{language=c++, numbers=left, tabsize=2, frame=single, breaklines=true, basicstyle=\ttfamily\tiny,
     numberstyle=\tiny\ttfamily, framexleftmargin=13mm, xleftmargin=12mm,keywordstyle=\color{blue}\bfseries,%
     commentstyle=\sffamily\color{red}}
  \lstset{emph={gcd,extendedEuclid,getCFrac},emphstyle=\color{MyGreen}}
  \begin{lstlisting}
  typedef BigInteger Integer;
  typedef LighterSternBrocot<Integer,int64_t> SB;
  typedef SB::Fraction Fraction;
  
  Fraction z( 643, 432 ); // classical instanciation
  SB::display( std::cout, z ); // z=z_3=[1,2,21,10]
  std::cout << std::endl;
  std::cout << "Nb nodes = " << SB::instance().nbFractions 
	    << std::endl; // 6 nodes 
  Fraction z2 = z.previousPartial(); // z_{n-1}
  SB::display( std::cout, z2 ); // z_2=[1,2,21]
  std::cout << std::endl;
  Fraction z1 = z.reduced( 2 ); // z_{n-2}
  SB::display( std::cout, z1 );  // z_1=[1,2]
  std::cout << std::endl;
  z.push_back( make_pair( 12, 4 ) ); // deeper fraction
  SB::display( std::cout, z ); // z=z_4=[1,2,21,10,12]
  // [Fraction f=7780/5227 u=12 k=4 [1,2,21,10,12] ]
  std::cout << std::endl;
  \end{lstlisting}

\end{frame}

\end{document}
