\documentclass[pdftex,francais]{beamer}

% Copyright 2004 by Till Tantau <tantau@users.sourceforge.net>.
%
% This file can be redistributed and/or modified under
% the terms of the GNU Public License, version 2.

%% \ifx\themename\undefined
%%   \def\themename{default}
%% \fi

\usetheme{lama}
%\usetheme{Madrid}
%\usecolortheme{crane}

\usepackage{multirow}
\usepackage[latin1]{inputenc}           %%%  
\usepackage[T1]{fontenc}                %%%
\usepackage[francais]{babel}            %%%

\usepackage{multimedia}
\usepackage{hyperref}
\usepackage{tikz}
\usepackage{listings}

%\newtheorem{theorem}{Théorème}

%\setbeamercovered{transparent}

\title[Digital surfaces in DGtal]{Digital surfaces in DGtal\\
     Topology module (since 0.5)
}

\author[J.-O. Lachaud]{Jacques-Olivier Lachaud}

\date{DGtal Meeting, june 2012}


\graphicspath{{Figures/},{Images/},{Graphs/}}

%%% \AtBeginSection[]
%%% {
%%%   \begin{frame}<beamer>
%%%     \frametitle{Plan}
%%%     \tableofcontents[currentsection] %,currentsubsection]
%%%   \end{frame}
%%% }

\input{macros}

\setbeamercolor{qcolorb}{fg={blue!20!black},bg={blue!15!white}}
\setbeamercolor{qcolorub}{fg={blue!10!black},bg={blue!30!white}}
\setbeamercolor{qcolorlb}{fg={blue!20!black},bg={blue!8!white}}
\setbeamercolor{qcolorulb}{fg={blue!10!black},bg={blue!40!white}}
\setbeamercolor{qcolorg}{fg={green!20!black},bg={green!15!white}}
\setbeamercolor{qcolorug}{fg={green!10!black},bg={green!30!white}}
\setbeamercolor{qcolorlg}{fg={green!20!black},bg={green!8!white}}
\setbeamercolor{qcolorulg}{fg={green!10!black},bg={green!40!white}}
\setbeamercolor{qcolorr}{fg={red!20!black},bg={red!15!white}}
\setbeamercolor{qcolorur}{fg={red!10!black},bg={red!30!white}}
\setbeamercolor{qcolorlr}{fg={red!20!black},bg={red!8!white}}
\setbeamercolor{qcolorulr}{fg={red!10!black},bg={red!40!white}}
\newenvironment{myblockbluish}[2]%
	       {\begin{beamerboxesrounded}[lower=qcolorb,upper=qcolorub,width=#1,shadow=true]{#2}}{\end{beamerboxesrounded}}
\newenvironment{myblocklbluish}[2]%
	       {\begin{beamerboxesrounded}[lower=qcolorlb,upper=qcolorulb,width=#1,shadow=true]{#2}}{\end{beamerboxesrounded}}
\newenvironment{myblockgreenish}[2]%
	       {\begin{beamerboxesrounded}[lower=qcolorg,upper=qcolorug,width=#1,shadow=true]{#2}}{\end{beamerboxesrounded}}
\newenvironment{myblocklgreenish}[2]%
	       {\begin{beamerboxesrounded}[lower=qcolorlg,upper=qcolorulg,width=#1,shadow=true]{#2}}{\end{beamerboxesrounded}}
\newenvironment{myblockredish}[2]%
	       {\begin{beamerboxesrounded}[lower=qcolorr,upper=qcolorur,width=#1
,shadow=true]{#2}}{\end{beamerboxesrounded}}
\newenvironment{myblocklredish}[2]%
	       {\begin{beamerboxesrounded}[lower=qcolorlr,upper=qcolorulr,width=#1,shadow=true]{#2}}{\end{beamerboxesrounded}}

%%%%%%%%%%%%%%%%%%%%%%%%%%%%%%%%%%%%%%%%%%%%%%%%%%%%%%%%%%%%%%%%%%%%%%%%%%%%%%%
%%%%%%%%%%%%%%%%%%%%%%%%%%%%%%%%%%%%%%%%%%%%%%%%%%%%%%%%%%%%%%%%%%%%%%%%%%%%%%%
%%%%%%%%%%%%%%%%%%%%%%%%%%%%%%%%%%%%%%%%%%%%%%%%%%%%%%%%%%%%%%%%%%%%%%%%%%%%%%%
\begin{document}

\newlength{\unquart}
\setlength{\unquart}{0.21\textwidth}

%------------------------------------------------------------------------------
\begin{frame}
  \titlepage
\end{frame}
%------------------------------------------------------------------------------

\section{Introduction}

%------------------------------------------------------------------------------
\begin{frame}%[allowframebreaks]
  \frametitle{Available in DGtal 0.5}
  
  \begin{enumerate}
  \item classical digital topology
    \begin{itemize}
    \item Arbitrary adjacencies in $\Z^n$, but also in subdomains
    \item Digital topology = couple of adjacencies (Rosenfeld)
    \item Object = Topology + Set
    \item Operations: neighborhoods, border, connectedness and connected
      components, decomposition into digital layers, simple points
    \end{itemize}

    \only<1>{
      \begin{tabular}{cc}
        \includegraphics[width=0.4\textwidth]{object-3d-18-6} & 
        \includegraphics[width=0.3\textwidth]{thinning-3d} \\
        Adjacencies &
        thinning in (6,26) \\
      \end{tabular}
    }

    \only<2-3>{
    \item cubical cellular topology
      \begin{itemize}
      \item cells, adjacent and incident cells, faces and cofaces
      \item signed cells, signed incidence, 
      \end{itemize}
    }
    \only<3>{
    \item digital surface topology
      \begin{itemize}
      \item surfels, surfel adjacency, surfel neighborhood
      \item surface tracking (normal, fast), contour tracking in $n$D
      \end{itemize}
    }
  \end{enumerate}
\end{frame}
%------------------------------------------------------------------------------


%------------------------------------------------------------------------------
\begin{frame}[squeeze]%[allowframebreaks]
  \frametitle{Package description}

  \begin{myblocklbluish}{\textwidth}{Should contain}
    \begin{itemize}
      \small
    \item classical digital topology {\em à la} Rosenfeld
    \item cartesian cellular topology
    \item digital surface topology {\em à la} Herman
    \item must be the base block of geometric algorithms
    \end{itemize}
  \end{myblocklbluish}
  \begin{myblocklgreenish}{\textwidth}{Examples}
    \begin{itemize}
      \small
    \item adjacencies, connected components, simple points, thinning
    \item cells, boundary operators, incidence, opening, closing
    \item contours, surfel adjacency, surface tracking
    \item topological invariants
    \end{itemize}
  \end{myblocklgreenish}
  \begin{myblocklredish}{\textwidth}{Location}
    \begin{itemize}
      \small
    \item \texttt{\{DGtal\}/src/DGtal/topology}
    \item \texttt{\{DGtal\}/src/DGtal/helpers}
    \item \texttt{\{DGtal\}/tests/topology}
    \end{itemize}
  \end{myblocklredish}

\end{frame}
%------------------------------------------------------------------------------


\end{document}
