% reference to figure
\newcommand{\RefFigure}[1]{Fig.\,\ref{#1}}
% reference to equation
\newcommand{\Equ}[1]{Equ\,(\ref{#1})}
% et al.
\newcommand{\etal}{{\it et~al.}}


% space of the real numbers
\newcommand{\R}{\ensuremath{\mathbb{R}}}
% space of the integer numbers
\newcommand{\Z}{\ensuremath{\mathbb{Z}}}
% Digitization process of step h (1)
\newcommand{\Dig}[1]{\ensuremath{\mathrm{Dig}_{#1}}}
% Family of shape
\newcommand{\SF}[0]{\ensuremath{\mathbb{F}}}
% Topological boundary of X (1).
\newcommand{\TB}[1]{\ensuremath{\partial #1}}
% Discrete geometric estimator of G (1)
\newcommand{\DGE}[1]{\ensuremath{E_{#1}}}
% Reference shape of digital object O (1) with grid step $h$ (2).
\newcommand{\RS}[2]{\ensuremath{R_{#1,#2}}}

% Digital contour.
\newcommand{\DC}{\ensuremath{C}}
% Continuous contour.
\newcommand{\CC}{\ensuremath{\mathcal{C}}}
% A point indexed by i (1) on the digital contour.
\newcommand{\PT}[1]{\ensuremath{\DC_{#1}}}
% A sequence of points indexed by i (1) on the digital contour.
\newcommand{\PTS}[2]{\ensuremath{\DC_{#1,#2}}}
% Predicate stating that the digital curve is a segment between indices 1 and 2
\newcommand{\SPRED}[2]{\ensuremath{S(#1,#2)}}
% ET logique
\newcommand{\AND}[0]{\ensuremath{\wedge}}
% OU logique
\newcommand{\OR}[0]{\ensuremath{\vee}}

% Tangent direction mapping of curve C (1)
\newcommand{\TGT}[1]{\ensuremath{\theta_{#1}}}
% Integral of squared curvature of curve C (1)
\newcommand{\ISC}[1]{\ensuremath{J[#1]}}
% Smallest possible tangent direction at constraint l (1)
\newcommand{\MinTD}[1]{\ensuremath{a_{#1}}}
% Largest possible tangent direction at constraint l (1)
\newcommand{\MaxTD}[1]{\ensuremath{b_{#1}}}
% Unknown tangent direction at constraint l (1)
\newcommand{\UnkTD}[1]{\ensuremath{t_{#1}}}

