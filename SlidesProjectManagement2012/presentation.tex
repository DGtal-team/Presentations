\documentclass[8pt]{beamer}
%epackage[french]{babel}
\usepackage[latin1]{inputenc}
\usepackage{listings}
\usepackage{times}
\usepackage{wasysym}
\usepackage[T1]{fontenc}

\usepackage{listings}

\definecolor{mongris}{gray}{0.8}           % definition couleur grise
\newcommand{\dd}{\footnotesize $\Diamond$}

\newcommand{\HH}{ \vspace{0.5pt}\hrule}
\newcommand{\round}[1]{\lceil #1 \rfloor}  % notation arrondi
\def\eme{$^{\textrm{{\`e}me}}$}                  % i {\`e}me
\def\num{n^{\circ}}                        % numero
\def\Num{N^{\circ}}                        % Numero
\def\sinc{\mathrm{sinc}}                   % sinus cardinal
\def\ere{$^{\textrm{{\`e}re}}$}                % {\`e}re
\def\er{$^{\textrm{{e}r}}$}                % {\`e}re
\def\eg{\emph{e.g.} }                      % e.g.
\def\ie{\emph{i.e.} }                      % i.e.
\def\etc{\emph{etc}}                       % etc
\def\cm{\,cm}                              % cm
\def\met{\,m}                              % m
\def\mm{\,mm}                              % mm
\def\deg{$^\circ$}                         % degres
\def\ud{\mathrm{d}}                        % pour dx dy ...


\def \R {{\Bbb R}}
\def \I {{\Bbb I}}
\def \H{{\Bbb H}}
\def \F {{\Bbb F}}
\def \S {{\Bbb S}}
\def \B {{\Bbb B}}
\def \Z {{\mathbb Z}}
\def \G {{\mathbb G}}
\def \L {{\mathcal{L}}}
\def \C {{\mathcal C}}
\def \P {{\mathcal P}}
\def \Q {{\mathcal Q}} 
\def \E{{\mathcal E}}
\def \D{{\mathcal D}}
\definecolor{mybluecolor}{RGB}{116,121,149}

\newcommand{\darky}[1]{{\usebeamercolor[fg]{block title example} #1}}
\newcommand{\myblue}[1]{{\color{mybluecolor}\aut{[#1]}}}

\newcommand{\ball}  {\ensuremath{B}}
\newcommand{\AMDR}{\operatorname{AMD}}
\newcommand{\AMD}{\operatorname{AMD}}

\newcommand{\MAset}{\ensuremath{\mathrm{A\!M}} }
\newcommand{\MAsetg}{\ensuremath{\MAset^g } }

\def \PS {{\aut{Planar-4-3-SAT}}}
\def \R {{\Bbb R}}
\def \I {{\Bbb I}}
\def \F {{\Bbb F}}
\def \S {{\Bbb S}}
\def \Z {{\mathbb Z}}
\def \L {{\mathcal{L}}}
\def \C {{\mathcal C}}
\def \P {{\mathcal P}}
\def \Q {{\mathcal Q}} 
\def \E{{\mathcal E}}
\def \D{{\mathcal D}}
\def \BD {{\bar{\mathcal{D}}}}
\def \etal {{\it et al.~}}
\def\arc{\mbox{arc}}
\definecolor{mongris}{gray}{0.8}          
\newcommand{\fup}[1]{\uparrow#1\uparrow}
\newcommand{\fdown}[1]{\downarrow#1\downarrow}
\newcommand{\sI}[1]{\overline{\tt #1}}
\newcommand{\iI}[1]{\underline{\tt #1}}
\newcommand{\e}[5]{#1 & #2 & #3 & #4 & #5 \\}
\newcommand{\eh}[5]{\text{#1} & \text{#2} &  \text{#3} &  \text{#4} & \text{#5}\\} 

\usepackage{beamerthemeliris2}
\useoutertheme{smoothbars}

\title[DGtal Meeting 2012]{DGtal Project Management}
\subtitle{\url{http://liris.cnrs.fr/dgtal}}

%\author{D. Coeurjolly}
\author[D. Coeurjolly]{David Coeurjolly}


 \newcommand{\fod}[2]{\multicolumn{2}{p{3.5cm}}{\emph{#1}\dotfill} &
      \multicolumn{2}{p{9cm}}{#2}\\}
    \newcommand{\fodt}[4]{\emph{#1} & {\footnotesize \textsl{#2}} & #3 & \small #4\\}
    % \newenvironment{ta}{\begin{tabular}{p{3.5cm}p{9cm}}}{\end{tabular}\\}
    \newenvironment{ta}{\begin{tabular}{crll}}{\end{tabular}\\}
    % \vfill


\newcommand{\aut}[1]{{\sc #1}}             % auteur en small capsu


%\institute%[XXX]
%{%
%
%  {\bf Laboratoire d'InfoRmatique en Image et Systèmes d'information} \\
%  { \scriptsize{
%  LIRIS UMR 5205 CNRS/INSA de Lyon/Université Claude Bernard Lyon 1/Université Lumiè%re Lyon 2/Ecole Centrale de Lyon\\
%  INSA de Lyon, bâtiment J. Verne\\
%  20, Avenue Albert Einstein - 69622 Villeurbanne cedex\\
%  \url{http://liris.cnrs.fr}}
%  }
%}



\graphicspath{{./Figures/}, {./../images/},{./Fig/}, {./ICPR2010/},{./Antoine/images/}; {./Images/}}


\begin{document}

\small






\begin{frame}[plain]
  \titlepage
\end{frame}


\begin{frame}
  \frametitle{DGtal team}
\small
  \begin{block}{Editorial Board\HH}
    \begin{itemize}
    \item Overall project management (discussions, objectives)
    \item  Prepares releases
    \item Organises DGtal meetings
      \item Board members are updated during DGtal meetings 
    \end{itemize}
 \emph{David Coeurjolly, Jacques-Olivier Lachaud}
  \end{block}

  \begin{block}{Package Managers\HH}
    \begin{itemize}
    \item In-charge of the package design (main concepts,
      relationships with other packages\ldots)
    \item Organise and check  package contributions (aka modules)
    \end{itemize}
  \end{block}

 \begin{block}{Contributors\HH}
    \begin{itemize}
    \item Add new functionnalities to DGtal (classes+concept+test
      files+documentation) as a new module in a DGtal package
    \end{itemize}
  \end{block}

We'd love to also have doc readers, build checkers, testers,...
\end{frame}


\begin{frame}
  \frametitle{DGtal Projects}

  \begin{block}{DGtal\HH}
    Main library (classes + unit tests + example programs + documentation files)
  \end{block}

\vspace{0.5cm}

  \begin{block}{DGtalTools\HH}
   Command line tools built using DGtal (multigrid shape generators,
   contour analyzers, file format conversion,...)
  \end{block}

\vspace{0.5cm}

  \begin{block}{DGtalScripts\HH}
    Scripts and template files to generate new classes/unit tests files
  \end{block}
\end{frame}

\begin{frame}
  \frametitle{How to submit a new feature ?}

  \begin{enumerate}
  \item Contact the associated Package Managers (or the Editorial
    board) 
    \begin{itemize}
    \item to get help on the package design and on DGtal API
    \item to interact with the package design if necessary
    \end{itemize}

  \item Prepare your module (classes+tests+doc) and submit your module
    as github ``pull-request''

  \item Available in the next release (at least, every 6 months)
  \end{enumerate}

\vspace{0.5cm}

For a new tool in  DGtalTools, contact the DGtalTools project manager
  (cf web site) 

\vspace{1cm}

\alert{DGtal help desk:}
\begin{itemize}
\item DGtal devel mailing lists (cf web site)
\item github pull-request/issue mechanisms when discussing on the code 
\end{itemize}

\end{frame}


\begin{frame}
  \frametitle{DGtal folder structure $\Leftrightarrow$ Packages  }

  \begin{description}

  \item[ \tt arithmetic/] 
  \item[ \tt base/] Tools (trace, timer, concept checking tools, basic functors\ldots)
  \item[ \tt geometry/] Geometrical package
    \begin{description}
        \item[ \tt curves/] algorithms/data structures on $1-d$ structures in dimension $n$
        \item[ \tt surfaces/] algorithms/data structures on $(n-1)-d$ structures in dimension $n$
        \item[ \tt volumes/] algorithms/data structures on $n-d$ structures in dimension $n$
        \item[ \tt helpers/] ...
        \item[ \tt tools/] ...
    \end{description}
  \item[ \tt helpers/]
  \item[ \tt images/]
  \item[ \tt io/] IO Package
   \begin{description}
        \item[ \tt boards/]
        \item[ \tt viewers/]
        \item[ \tt colormaps/]
        \item[ \tt readers/] ...
        \item[ \tt writers/] ...
   \end{description}
  \item[ \tt kernel/] Kernel
   \begin{description}
        \item[ \tt domains/]
        \item[ \tt sets/]
   \end{description}  \item[ \tt math/]
  \item[ \tt shapes/]
    \item[ \tt topology/] 
  \end{description}
\end{frame}

\begin{frame}
  \frametitle{DGtalTools folder structure}

   \begin{description}
        \item[ \tt 2dContourTools/] image to contour, ...
        \item[ \tt estimators/] multigrid evaluation of  DGtal estimator
        \item[ \tt shapeGenerator/] generate multigrid shape
        \item[ \tt voltools/] CLI tools to handle .vol files
        \item[ \tt volumetric/] DT, connected component, homotopic thinner....
   \end{description}  

   \begin{block}{Idea\HH}
     \begin{itemize}
     \item Weaker  structure compared to DGtal
     \item Tools focused on a specific analysis
     \item Good spot to test some ideas before promoting them to DGtal
     \end{itemize}
     
   \end{block}

\end{frame}

\begin{frame}
  \frametitle{Websites}

<Demo website, github, cdash>
\end{frame}

\begin{frame}
  \frametitle{Conclusion}
\end{frame}

\end{document}



%Sample.dat -> GRidCurve -> board (GC, GC::Arrows, GC::Inci)

%Image -> Set ->  DT -> board

%Image -> contour (KSpace, track) -> GC -> estimateur (longueur)

%Shape -> Digitizer -> Contour -> GC -> Estimateur

% Shape -> surface -> viewer

% Shape -> Surface -> slice -> GC -> longueur (viewer Segm.DSS)
